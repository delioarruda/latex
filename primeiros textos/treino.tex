% pacote de formatção de papel e tipo de documento
\documentclass[a4paper,12pt]{article}
%pacote para referenciar margens do documento
\usepackage[top=2cm,bottom=2cm,left=2.5cm,right=2.5cm]{geometry}
%pacote com o tipo de codificação usada
\usepackage[utf8]{inputenc}
%pacote para simbolos e fontes
\usepackage {amsmath, amsfonts, amssymb}
%para decalrar o operador matemático
\DeclareMathOperator{\sen}{sen}
\DeclareMathOperator{\tg}{tg}
\DeclareMathOperator{\cossec}{cossec}
% define o título
\author{Délio de Arruda Almeida}
%definir o titulo
\title{Educação Estatística}
\begin{document}
   % gera o título
   \maketitle
   % insere o índices
   \tableofcontents

   \section{Equação Polinômial do 2º Grau}
     É uma equação do tipo $ax^2+bx+c=0$ com $a\neq 0$ será chamada de        equação polinomial do 2º grau.

   A solução dessa equação é dada por: $$x=\frac{-b\pm\sqrt[]{b^2-4ac}}{2a}$$

   \begin{center}
   para centralizar 
   \end{center}
   \begin{flushright}
   para alinhar a direita
   \end{flushright}
   \begin{flushleft}
   para alinhar a esquesda
   \end{flushleft}

   Bem, \textit{aqui inicia} \textit{\textbf{meu adorável}} artigo.

   \underline{para deixar sublinhado}

   \textbf{\textit{\underline{para dar os três efeitos basta aninhar}}}
   \section{Tchau Mundo}
   \ldots{} e aqui ele termina.
   \begin{enumerate}
       \item Primeira Questão
          \begin{enumerate}
             \item Primeira acertiva da questão
                \begin{enumerate}
                   \item terceito nível
                \end{enumerate}
             \item Segunda acertiva da questão
          \end{enumerate}

       \item Segunda Questão
       \item Terceira Questão 
   \end{enumerate}
   \begin{itemize}
      \item primerio item da lista
       não é item da lista
      \item segundo item da lista
          \begin{itemize}
            \item primteiro item da sublista
            $a \cdot b$ $$ a\times b $$
          \end{itemize}
   \end{itemize}
   isso é uma fração na formatação da linha: $\frac{a}{b}$ 
   
   isso é uma fração fora da linha: $\dfrac{a}{b}$
   
   potência: $a^{(b+c)}$
   
   sub escritos: $a_{bacaxi}$
   
   %para inserir espaço entre os itens digite \, para inserir sibolos usados pelo latex use \ exemplo: \}   
   Sejam os conjuntos: $ A = \{ a,\,b,\,c,\,d\}$
   
   $$ B= \{ x\in \mathbb{R} \,\mid \,-2 \leq \leqslant x < 4\geqslant \geq 0 \}  $$ 
   % nas notações de conjuntos A - B também pode ser representado assim
   $$A \setminus B$$
   
   $\subset \not\subset \ntriangleleft \subsetneq \varsubsetneq\nsubseteq$ \newline $7 \not\in \{ x \in \mathbb{N} \, \mid \,x \textrm{ é par}\}$  
   \begin{enumerate}
      \item Seja a função $f:\mathbb{R} \to \mathbb{R}$ definida por $f(x)= \dfrac{1}{2}x^2-2x + 1$.
      \begin{enumerate}
         \item Esboce o gráfico da função.
         \item $x \mapsto \dfrac{1}{2}x^2-2x + 1$.
         $$f(x)=   
   \begin{cases}
      x^2 - 1; \,\textrm{se } x \geq 1\\
      x-3;\,\textrm{se }-1 \leq x < 1\\
      2x + 1; \,\textrm{se } x>1
   \end{cases}$$
         \item $f(x)= \log_2x + \ln x $
         \item $f(x) = \cos x$.
         \item $f(x) = \sin x$.
         \item $f(x) = \textrm{sen}\, x$.
         \item $f(x) = \sen \left( x- \dfrac{\pi}{2}\right)$.
         \item $f(x) = \sen \left[ x- \dfrac{\pi}{2}\right]$.
         
          \item $f(x) = \sen \left\{ x- \dfrac{\pi}{2}\right\}$.
      \end{enumerate}
   \end{enumerate}     
   
   
   
\end{document}