% pacote de formatção de papel e tipo de documento
\documentclass[a4paper,12pt]{article}
%pacote para referenciar margens do documento
\usepackage[top=2cm,bottom=2cm,left=2.5cm,right=2.5cm]{geometry}
%pacote com o tipo de codificação usada
\usepackage[utf8]{inputenc}
%pacote para simbolos e fontes
\usepackage {amsmath, amsfonts, amssymb}
% pacote para inserir figuras
\usepackage{graphicx}
%pacote para inserir a figura extamente no local, ignorando a formatação do texto e espaçamento
\usepackage{float}
\usepackage[portuguese]{babel}
%\setlength{\voffset}{-1.0cm} \setlength{\hoffset}{-0.5cm}
%para decalrar o operador matemático
\DeclareMathOperator{\sen}{sen}
\DeclareMathOperator{\tg}{tg}
\DeclareMathOperator{\cossec}{cossec}
%definir novo comando
\newcommand{\limite}{\displaystyle\lim}
\newcommand{\integral}{\displaystyle\int}
\newcommand{\somatorio}{\displaystyle\sum}
% define o título
\author{Délio de Arruda Almeida}
%definir o titulo
\title{Educação Estatística}
\begin{document}
   % gera o título
   \maketitle
   % insere o índices
   \tableofcontents

   \section{Equação Polinômial do 2º Grau}
     É uma equação do tipo $ax^2+bx+c=0$ com $a\neq 0$ será chamada de        equação polinomial do 2º grau.

   A solução dessa equação é dada por: $$x=\frac{-b\pm\sqrt[]{b^2-4ac}}{2a}$$

   \begin{center}
   para centralizar 
   \end{center}
   \begin{flushright}
   para alinhar a direita
   \end{flushright}
   \begin{flushleft}
   para alinhar a esquesda
   \end{flushleft}

   Bem, \textit{aqui inicia} \textit{\textbf{meu adorável}} artigo.

   \underline{para deixar sublinhado}

   \textbf{\textit{\underline{para dar os três efeitos basta aninhar}}}
   \section{Tchau Mundo}
   \ldots{} e aqui ele termina.
   \begin{enumerate}
       \item Primeira Questão
          \begin{enumerate}
             \item Primeira acertiva da questão
                \begin{enumerate}
                   \item terceito nível
                \end{enumerate}
             \item Segunda acertiva da questão
          \end{enumerate}

       \item Segunda Questão
       \item Terceira Questão 
   \end{enumerate}
   \begin{itemize}
      \item primerio item da lista
       não é item da lista
      \item segundo item da lista
          \begin{itemize}
            \item primteiro item da sublista
            $a \cdot b$ $$ a\times b $$
          \end{itemize}
   \end{itemize}
   isso é uma fração na formatação da linha: $\frac{a}{b}$ 
   
   isso é uma fração fora da linha: $\dfrac{a}{b}$
   
   potência: $a^{(b+c)}$
   
   sub escritos: $a_{bacaxi}$
   
   %para inserir espaço entre os itens digite \, para inserir sibolos usados pelo latex use \ exemplo: \}   
   Sejam os conjuntos: $ A = \{ a,\,b,\,c,\,d\}$
   
   $$ B= \{ x\in \mathbb{R} \,\mid \,-2 \leq \leqslant x < 4\geqslant \geq 0 \}  $$ 
   % nas notações de conjuntos A - B também pode ser representado assim
   $$A \setminus B$$
   
   $\subset \not\subset \ntriangleleft \subsetneq \varsubsetneq\nsubseteq$ \newline $7 \not\in \{ x \in \mathbb{N} \, \mid \,x \textrm{ é par}\}$  
   \begin{enumerate}
      \item Seja a função $f:\mathbb{R} \to \mathbb{R}$ definida por $f(x)= \dfrac{1}{2}x^2-2x + 1$.
      \begin{enumerate}
         \item Esboce o gráfico da função.
         \item $x \mapsto \dfrac{1}{2}x^2-2x + 1$.
         $$f(x)=   
   \begin{cases}
      x^2 - 1; \,\textrm{se } x \geq 1\\
      x-3;\,\textrm{se }-1 \leq x < 1\\
      2x + 1; \,\textrm{se } x>1
   \end{cases}$$
         \item $f(x)= \log_2x + \ln x $
         \item $f(x) = \cos x$.
         \item $f(x) = \sin x$.
         \item $f(x) = \textrm{sen}\, x$.
         \item $f(x) = \sen \left( x- \dfrac{\pi}{2}\right)$.
         \item $f(x) = \sen \left[ x- \dfrac{\pi}{2}\right]$.
         
          \item $f(x) = \sen \left\{ x- \dfrac{\pi}{2}\right\}$.
      \end{enumerate}
   \end{enumerate}     
   %a partir daqui seguiremos pelo versionamento git
   %para exibir matriz
   \begin{enumerate}
    \item$\begin{bmatrix}
      1 & 10 & -5 \\
      6&7&8
   \end{bmatrix}$  
   \item $ \begin{pmatrix}
      1 & 10 & -5 \\
      6&7&8 \\ 9&3&2
      \end{pmatrix}$
      \item $\begin{vmatrix}
      1 & 10 & -5 \\
      6&7&8
      \end{vmatrix}$ 
   \end{enumerate}
   \begin{enumerate}
      \item Consider a matriz $$M= \begin{bmatrix}
      1&10& -5\\6&7&8\\3&21&12
      \end{bmatrix}$$Calcule o que for solicitado abaixo.
      \begin{enumerate}
         \item $\det M$
         \item $M^{-1}$
         \item $M^T$
      \end{enumerate}
   \end{enumerate}
   \begin{enumerate}
      \item Considere a matriz $m\times n$ dada por $\begin{bmatrix}
      a_{11} & a_{12} & a_{13} & \cdots & a_{1n} \\ a_{21} & a_{22} & a_{23} & \cdots & a_{2n} \\ a_{31} & a_{32} & a_{33} & \cdots & a_{3n} \\ \vdots & \vdots & \vdots & \ddots & \vdots \\ a_{m1} & a_{m2} & a_{m3} & \cdots & a_{mn}  
      \end{bmatrix}$
   \end{enumerate}
   \begin{enumerate}
      \item Determine $x$, $y$, $z$ na equação: $$\begin{bmatrix}
      1& -2&4\\5&2& -2\\ 6&1&8
      \end{bmatrix}
      \begin{bmatrix}
         x \\ y \\ z
      \end{bmatrix}=
      \begin{bmatrix}
        2\\ 10\\ 6
      \end{bmatrix}$$
   \end{enumerate}
   %notação de segmento
   \begin{enumerate}
      \item Seja o segmento $\overline{AB}$. A partir dele podemos definir o segmento orientado $\overrightarrow{AB}$  e o segmento orientado $\overleftarrow{BA}$.
      %notação de vetor
      \item Seja o vetor $\vec{u}$.
      \item Sejam os vetores $\vec{u}=(1;\,-1;\,2) \text{ e } \vec{v} = (2;\,5;\, -4)$.
      Calcule o seguinte:
      \begin{enumerate}
         \item $\vec{u} \cdot \vec{v}$
         \item $\vec{u} \times \vec{v}$
         %outra forma de representar
         \item Errado: $<\vec{u}\,\vec{v}>$. correto: $\langle\vec{u},\,\vec{v}\rangle$
         %notação para modulo com barra dupla
         \item $\|\vec{u}\|$
         \item Representação errada do segmento orientado em modulo $\|\overrightarrow{AB}\|$
         \item Representação correta do segmento orientado em modulo $\left\|\overrightarrow{AB}\right\|$
         \item Representação de vetores ortogonais: $$\vec{u} \perp \vec{v}$$
         \item Letras gregas $$\alpha$$ $$\beta$$
         \item Sejam os vetores $\vec{u} = {x_0;\, y_0;\, z_0}$ e $\vec{v}={x_1;\, x_1;\, z_1}$. Temo que:$$
         \vec{u} \times \vec{v} = \begin{vmatrix}
         \vec{i} & \vec{j} & \vec{k}\\ x_0 & y_0 & z_0 \\ x_1 & y_1 & z_1
         \end{vmatrix}$$
         
      \end{enumerate}
   \end{enumerate}
   \section{Aula 8 (Notações de Claculo)}
   \begin{enumerate}
   %o limite ficara estabelecido ao lado do operador
      \item $\lim_{x \to 1} \dfrac{x^2 - 1}{x - 1}$
      % com o $$ o limite será estabelecido na forma usual
       \item $$\lim_{x \to 1} \dfrac{x^2 - 1}{x - 1}$$
      % com o \displaystyle o limite será  forçado a ser exibido na forma usual
      \item $\displaystyle\lim_{x \to 1} \frac{x^2 - 1}{x - 1}$
      %depois de estabelecido o novo comando \newcommand{\limite}{\displaystyle\lim} faremos:
      \item $\limite_{x \to 1} \frac{x^2 - 1}{x - 1}$
      \item Derivadas:
         \begin{enumerate}
            \item $f'$ \item $f''$ \item $f'''$ \item $f^{(v)}$
         \end{enumerate}
      \item Seja a função definida por $f(x)=x^2 - \sqrt{x}$. Calcule as derivadas abaixo.
      \begin{enumerate}
         \item $\dfrac{df}{dx}$
         \item $\dfrac{d^2f}{dx^2}$
         \item $\dfrac{d^5f}{dx^5}$
         \item $\dfrac{d^3f}{dx^3}$
      \end{enumerate}
      \item Seja a função definida por $f(x, \, y)=yx^2 - \sqrt{x} + y^3$. Calcule as derivadas abaixo.
      \begin{enumerate}
         \item $\dfrac{\partial f}{\partial x}$
         \item $\dfrac{\partial^2f}{\partial x^2}$
         \item $\dfrac{\partial^5f}{\partial x^5}$
         \item $\dfrac{\partial^3f}{\partial x^3}$
         \item $\dfrac{\partial }{\partial x}\left(\dfrac{\partial f}{\partial y} \right)$
      \end{enumerate}
      \item Calcule as integrais abaixo.
      \begin{enumerate}
         \item $\int_1^5 x^2\cos x \, dx$
         \item $$\int_1^5 x^2\cos x \, dx$$
         \item $\displaystyle\int_1^5 x^2\cos x \, dx$
         \item $\integral_1^5 x^2\cos x \, dx$
      \end{enumerate}
      \item Calcule as integrais com somatórios abaixo.
      \begin{enumerate}
         \item $\sum_{i=1}^n \integral_0^{\infty} x^i\,dx$
         \item $$\sum_{i=1}^n \integral_0^{\infty} x^i\,dx$$
         \item $\displaystyle\sum_{i=1}^n \int_0^{\infty} x^i\,dx$
         \item $\displaystyle\sum_{i=1}^n $ $\int_0^{\infty} x^i\,dx$
         \item $\somatorio_{i=1}^n \integral_0^{\infty} x^i\,dx$
      \end{enumerate}
   \end{enumerate}
   
   \section{Aula9 (Como inserir figuras no Latex)}
   
   \begin{enumerate}
      \item Calcule o valor da $x$ na figura \ref{triangulo}.%com essa referencia ele vai inserir o numero da figura 
      \begin{figure}[!htb]
      %como estou usanto scale=0.5 sera inserida a metade do tamanho da figura, como esta na mesma pasta não preciso colocar o caminho, apenas o nome do arquivo
         \includegraphics[scale=0.5]{triangulo.jpg}
         \caption{triângulo maior.}
         \label{triangulo}
      \end{figure}
      \item Usando o pacote float\\
      \begin{figure}[H]
      % o pacote float vai forçar a imagem a ficar extamente onde está
         \centering %para centralizar
         \includegraphics[scale=0.2]{triangulo.jpg}
         \caption{triângulomenor .} %para inserir legenda na figura
         \label{triangulo menor}
      \end{figure}
       \item Usando a minha imagem em pasta diferente\\
      \begin{figure}[H]
      % o pacote float vai forçar a imagem a ficar extamente onde está
         \centering %para centralizar
         \includegraphics[scale=0.1]{imagens/delio.jpg}
         \caption{minha foto .} %para inserir legenda na figura
         \label{triangulo menor}
      \end{figure}
   \end{enumerate}      
   
\end{document}